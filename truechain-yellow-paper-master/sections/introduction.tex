\section{Introduction}

With the surging popularity of cryptocurrencies, blockchain technology has caught attention from both industry and academia.
One can think blockchain as a shared computing environment involving peers to join and quit freely, with the premis for a commonly
agreed consensus protocol. The decentralized nature of blockchain, together with transaction transparency, autonomy, immutability,
are critical to crypocurrencies, drawing the baseline for such systems. However top earlier-designed cryptocurrencies,
such as Bitcoin\cite{nakamoto2008bitcoin} and Ethereum\cite{buterinethereum}, have been widely recognized unscalable in terms
of transaction rate and are not economically viable as they require severe energy consumptions and computation power.

With the demand of apps and platforms using public blockchain growing in real world, a viable protocol that enables higher
transaction rate is a main focus on new systems. For example, consider a generic public chain that could host CPU intensive
peer to peer gaming applications with a very large userbase. In such a chain, if it also hosts ICOs as well as digital advertisement
applications, we currently expect a huge delay in transaction confirmation times.

There are other models like delegated mechanism of Proof of Stake and Permissioned Byzantine Fault Tolerant protocols.
The BFT protocol ensures safety as long as only one third of the actors in the system are intentionally/unintentionally malacious adversaries,
at a time. This is a really good mechanism, however a BFT chain alone has a problem with scalability and pseudo-decentralization.
The Proof of Stake protocol with a small number of validators could although facilitate high throughput, but the system in itself is
highly dependent on a few stakeholders to make the decisions on inclusion and exclusion of delegates. Moreover, there is no transparency
without Merkel trees and this type of system could always suffer from nothing-at-stake paradox.

In this paper, we propose TrueChain, a hybrid protocol\cite{pass2017hybrid} of PBFT\cite{castro1999practical} and POW consensus.
The POW consensus ensures incentivization and committee selection while the PBFT layer acts as a highly performant consensus
with capabilities like instant finality with high throughput, transaction validation, rotating committee for fair trade economy
and a compensation infrastructure to deal with non-uniform infrastructure. The nature of hybrid protocol allows it to tolerate
corruptions at a maximum of about one third of peer nodes.
